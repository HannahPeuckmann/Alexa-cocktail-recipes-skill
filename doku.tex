\documentclass[12pt,letterpaper]{article}
\usepackage[utf8]{inputenc}
\usepackage{amsmath}
\usepackage{amsfonts}
\usepackage{amssymb}
\usepackage{titlesec}
\usepackage{makeidx}
%set page header and footer
\usepackage{fancyhdr}
\usepackage{dialogue}
\pagestyle{fancy}
\lhead{Sex On The Beach}
\rhead{\today}

\usepackage[left=4cm,right=4cm,top=4cm,bottom=4cm]{geometry}

\titlelabel{\thetitle\enspace}

\begin{document}
\title{Sex On The Beach - Ein Alexa Skill}
\author{Sara Derakhshani Hannah Peuckmann}
\maketitle
\thispagestyle{fancy}

Zusammenfassung hier hin, oder lieber neue Seite, Index und zusammenfassung als eigenne section?

\section{Projektziele, Anforderungsdefinition}
\subsection{Benutzerpersonae}
\subsection{Beispieldialoge}
\begin{dialogue}
	\speak{USER} Open sex on the beach.
	\speak{ALEXA} Welcome to sex on the beach.
	\speak{USER} Give me a cocktail Idea.
	\speak{ALEXA} I think a Dark and Stormy sounds good. Should I list the ingredients for you?
	\speak{USER} Yes.
	\speak{ALEXA} 2 ingredients are needed for a Dark and Stormy: Dark Rum, Ginger Beer.
	\speak{USER} Give me the instructions for Dark and Stormy.
	\speak{ALEXA} The instructions for dark and stormy are the following:
	In a highball glass filled with ice add 6cl dark rum and top with ginger beer. Garnish with lime wedge.
	\speak{USER} Which glass do I need?.
	\speak{ALEXA} Dark and stormy is served in a Highball glass.
	\speak{User} Bye
	\speak{Alexa} Enjoy your cocktail.
\end{dialogue}

\begin{dialogue}
	\speak{USER} Open sex on the beach
	\speak{ALEXA} Welcome to sex on the beach.
	\speak{USER} Give me the ingredients for a Ipanema.
	\speak{ALEXA} You need 4 ingredients to make a ipanema:
	Cachaca, Lemon Juice, Agave Syrup, Champagne.
	\speak{USER} search for the instructions.
	\speak{ALEXA} Here are the instructions for making a ipanema: Add the cachaca, lemon juice and syrup to your boston glass.
	Add ice and shake until ice cold. Pour into a chilled flute and top-up with Champagne.
	\speak{USER} search for white russian.
	\speak{ALEXA} what do you want me to look up, instructions, ingredients or both?
	\speak{USER} both.
	\speak{ALEXA} You need 3 ingredients to make a white Russian: Vodka, Coffee liqueur, Light cream. Pour vodka and coffee liqueur over ice cubes in an old-fashioned glass. Fill with light cream and serve.
	\speak{USER} search for cocktails with coffee liqueur
	\speak{ALEXA} There are probably a lot of cocktails with coffee liqueur ,
	tell me a second ingredient to narrow down the search
	\speak{USER} Milk
	\speak{ALEXA} Ups, I don't know such exotic cocktails with coffee liqueur and milk
\end{dialogue}

\begin{dialogue}
	\speak{USER} Sex on the beach
	\speak{ALEXA} Welcome to sex on the beach.
	\speak{USER} I want to make a gin basil smash.
	\speak{ALEXA} What do you want me to look up, instructions , ingredients or both?
	\speak{USER} Ingredients.
	\speak{ALEXA} You need 4 ingredients to make a gin basil smash: Gin, Lemon Juice, Sugar Syrup, Basil .
	\speak{USER} How much gin do i need?
	\speak{ALEXA} You need 6 cl gin for a gin basil smash.
	\speak{USER} What are the instructions?
	\speak{ALEXA} Here are the instructions for making a gin basil smash: You need 6 cl Gin, 2 cl Lemon Juice, 2 cl Sugar Syrup, Whole Basil. Muddle Basil leaves (~ 10) with Suggar Syrup in a shaker. Add Gin an Lemon Juice. Filter and serve in a tumbler with ice.
	\speak{USER} Tell me something about gin.
	\speak{ALEXA} Gin is a distilled alcoholic drink that derives its predominant flavour from juniper berries (Juniperus communis)..Gin is one of the broadest categories of spirits, all of various origins, styles, and flavour profiles, that revolve around juniper as a common ingredient.
	
\end{dialogue}

\begin{dialogue}
	\speak{USER} Sex on the beach
	\speak{ALEXA} Welcome to sex on the beach.
	\speak{USER} Give me a cocktail suggestion.
	\speak{ALEXA} I think a Avalon sounds good. Do you want to hear the ingredients list?
	\speak{USER} No.
	\speak{ALEXA} Okay, maybe next time!
	\speak{USER} Give me a cocktail suggestion.
	\speak{ALEXA} You can't do anython wrong with a Bramble Should I list the ingredients for you?
	\speak{USER} Yes.
	\speak{ALEXA} You need 4 ingredients to make a Bramble: Gin, lemon juice, Sugar syrup, Creme de Mure.
	\speak{USER} What is creme de mure?
	\speak{ALEXA} Home-made blackberry liqueur, a warming, sweet drink for the winter months.
	\speak{USER} Yes.
	\speak{ALEXA} You need 4 ingredients to make a Bramble: Gin, lemon juice, Sugar syrup, Creme de Mure.
	\speak{USER} What is creme de mure?
	\speak{ALEXA} Home-made blackberry liqueur, a warming, sweet drink for the winter months.
	
\end{dialogue}

\begin{dialogue}
	\speak{USER} Sex on the beach
	\speak{ALEXA} Welcome to sex on the beach.
	\speak{USER} I want to make a cocktail without alcohol.
	\speak{ALEXA} Which ingredient do you want in your non alcoholic cocktail?
	\speak{USER} Milk.
	\speak{ALEXA} I know 10 cocktails with milk and no alcohol. I could give you three examples?
	\speak{USER} Yes, please.
	\speak{ALEXA} Banana Milk Shake, Sweet Bananas, Frappé.
	\speak{USER} Give me the ingredients for a frappe.
	\speak{ALEXA} 3 ingredients are needed for a frappe: Coffee, Milk, Sugar.
	\speak{USER} How much sugar?
	\speak{ALEXA} You need 1-2 tsp sugar for a frappe.
	\speak{USER} Which glass do i need?
	\speak{ALEXA} For a frappe a Highball Glass is the right choice.
	\speak{USER} What are the instructions?
	\speak{ALEXA} The instructions for frappe are the following: You need 1/2 cup black Coffee, 1/2 cup Milk, 1-2 tsp Sugar. Mix together. Blend at highest blender speed for about 1 minute. Pour into a glass and drink with a straw. Notes: This works best if everything is cold (if you make fresh coffee, mix it with the milk and let it sit in the fridge for 1/2 hour. If it is not frothy, add more milk, or even just some more milk powder. The froth gradually turns to liquid at the bottom of the glass, so you will find that you can sit and drink this for about 1/2 hour, with more iced coffee continually appearing at the bottom. Very refreshing.
	
\end{dialogue}

\section{Projektorganisation}
\subsection{Aufgabenverteilung}
\subsection{Planungsdokumentation}

\section{Entwurf des Systems, Dokumentation}
\subsection{Entwicklungsumgebung, Dokumentation der Intents}

\section{Projektabschluss, Evaluation}
\subsection{Versuchsanordnung}
\subsection{Auswertung}
\subsection{Ausblick}
\end{document}