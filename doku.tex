\documentclass[12pt,letterpaper]{article}
\usepackage[utf8]{inputenc}
\usepackage{amsmath}
\usepackage{amsfonts}
\usepackage{changepage}
\usepackage{amssymb}
\usepackage{titlesec}
\usepackage{makeidx}
%set page header and footer
\usepackage{fancyhdr}
\usepackage{dialogue}
\usepackage{attrib}
\pagestyle{fancy}
\lhead{Sex On The Beach}
\rhead{\today}

\usepackage[left=4cm,right=4cm,top=4cm,bottom=4cm]{geometry}

\titlelabel{\thetitle\enspace}

\begin{document}
\title{Sex On The Beach - Ein Alexa Skill}
\author{Sara Derakhshani Hannah Peuckmann}
\maketitle
\thispagestyle{fancy}

Sex on the beach - Im Ramen des Seminars Dialogmodelierung ist als abschließendes Projekt ein
Alexa skill entstanden, der Alexa zur Expertin werden lässt was Cocktails angeht.
Die grundlegende Funktionalität des Skills besteht darin, dass Alexa für einen Cocktail Auskunft geben kann,
über das Rezept, wobei zwischen Zutaten und Anleitung unterschieden wird, und über das Glas in dem der Cocktail
üblicherweise serviert wird. Benutzer*Innen können auch Informationen erhalten über in Rezepten enthaltene Zutaten,
Alexa gibt dann eine kleine Zusammenfassung was die Zutat ist. Die Mengenangaben der Zutaten können ebenfalls noch einmal extra erfragt werden pro Zutat.
Darüber hinaus kann Alexa Cocktails herraussuchen, die, von Benutzer*Innen
gegebene, Zutaten enthalten, wobei auch nach nichtalkoholischen Cocktails gefiltert werden kann. 
Benutzer*Innen können sich auch von Alexa einen willkürlichen Cocktail vorschlagen lassen.
Wärend der Entwicklung haben diese Funktionalität immer besser ineinander gegriffen, so das am Ende ein flüssiger Dialog
zwischen Benutzer*Innen und Alexa entstehen kann.

\section{Projektziele, Anforderungsdefinition}
\subsection{Benutzerpersonae}
Die zu erwartenden Benutzer*Innen des Skills sind Menschen einer breiten Altersspanne ab 18 Jahre, die Datenbank aus der die Rezepte bezogen werden erhält zwar auch nicht alkoholische Cocktails, die meisten Rezepte enthalten jedoch hochprozentigen Alkohol. Aufgrund dessen ist der Skill für volljährige Benutzer*Innen vorgesehen. Des weiteren  brauchen Benutzer*Innen ausreichende Kenntnisse der englischen Sprache, da der Skill auf englisch ist.


\subsection{Beispieldialoge}
\begin{dialogue}
	\speak{USER} Open sex on the beach.
	\speak{ALEXA} Welcome to sex on the beach.
	\speak{USER} Give me a cocktail Idea.
	\speak{ALEXA} I think a Dark and Stormy sounds good. Should I list the ingredients for you?
	\speak{USER} Yes.
	\speak{ALEXA} 2 ingredients are needed for a Dark and Stormy: Dark Rum, Ginger Beer.
	\speak{USER} Give me the instructions for Dark and Stormy.
	\speak{ALEXA} The instructions for dark and stormy are the following: You need 5 cl Dark Rum, 10 cl Ginger Beer. In a highball glass filled with ice add 6cl dark rum and top with ginger beer. Garnish with lime wedge.
	\speak{USER} Which glass do I need?.
	\speak{ALEXA} Dark and stormy is served in a Highball glass.
	\speak{User} Bye
	\speak{Alexa} Enjoy your cocktail.
\end{dialogue}

\begin{dialogue}
	\speak{USER} Open sex on the beach
	\speak{ALEXA} Welcome to sex on the beach.
	\speak{USER} Give me the ingredients for a Ipanema.
	\speak{ALEXA} You need 4 ingredients to make a ipanema:
	Cachaca, Lemon Juice, Agave Syrup, Champagne.
	\speak{USER} search for the instructions.
	\speak{ALEXA}The instructions for ipanema are the following: You need 25 ml Cachaca, 15 ml Lemon Juice, 10 ml Agave Syrup, top up with Champagne. Add the cachaca, lemon juice and syrup to your boston glass. Add ice and shake until ice cold. Pour into a chilled flute and top-up with Champagne.
	\speak{USER} search for white russian.
	\speak{ALEXA} what do you want me to look up, instructions, ingredients or both?
	\speak{USER} both.
	\speak{ALEXA} The instructions for white Russian are the following: You need 2 oz Vodka, 1 oz Coffee liqueur, some Light cream. Pour vodka and coffee liqueur over ice cubes in an old-fashioned glass. Fill with light cream and serve.
	\speak{USER} search for cocktails with coffee liqueur
	\speak{ALEXA} There are probably a lot of cocktails with coffee liqueur ,
	tell me a second ingredient to narrow down the search
	\speak{USER} Milk
	\speak{ALEXA} Ups, I don't know such exotic cocktails with coffee liqueur and milk
\end{dialogue}

\begin{dialogue}
	\speak{USER} Sex on the beach
	\speak{ALEXA} Welcome to sex on the beach.
	\speak{USER} I want to make a gin basil smash.
	\speak{ALEXA} What do you want me to look up, instructions , ingredients or both?
	\speak{USER} Ingredients.
	\speak{ALEXA} You need 4 ingredients to make a gin basil smash: Gin, Lemon Juice, Sugar Syrup, Basil .
	\speak{USER} How much gin do i need?
	\speak{ALEXA} You need 6 cl gin for a gin basil smash.
	\speak{USER} What are the instructions?
	\speak{ALEXA} Here are the instructions for making a gin basil smash: You need 6 cl Gin, 2 cl Lemon Juice, 2 cl Sugar Syrup, Whole Basil. Muddle Basil leaves (~ 10) with Suggar Syrup in a shaker. Add Gin an Lemon Juice. Filter and serve in a tumbler with ice.
	\speak{USER} Tell me something about gin.
	\speak{ALEXA} Gin is a distilled alcoholic drink that derives its predominant flavour from juniper berries (Juniperus communis)..Gin is one of the broadest categories of spirits, all of various origins, styles, and flavour profiles, that revolve around juniper as a common ingredient.
	
\end{dialogue}

\begin{dialogue}
	\speak{USER} Sex on the beach
	\speak{ALEXA} Welcome to sex on the beach.
	\speak{USER} Give me a cocktail suggestion.
	\speak{ALEXA} I think a Avalon sounds good. Do you want to hear the ingredients list?
	\speak{USER} No.
	\speak{ALEXA} Okay, maybe next time!
	\speak{USER} Give me a cocktail suggestion.
	\speak{ALEXA} You can't do anython wrong with a Bramble Should I list the ingredients for you?
	\speak{USER} Yes.
	\speak{ALEXA} You need 4 ingredients to make a Bramble: Gin, lemon juice, Sugar syrup, Creme de Mure.
	\speak{USER} What is creme de mure?
	\speak{ALEXA} Home-made blackberry liqueur, a warming, sweet drink for the winter months.
	\speak{USER} Yes.
	\speak{ALEXA} You need 4 ingredients to make a Bramble: Gin, lemon juice, Sugar syrup, Creme de Mure.
	\speak{USER} What is creme de mure?
	\speak{ALEXA} Home-made blackberry liqueur, a warming, sweet drink for the winter months.
	
\end{dialogue}

\begin{dialogue}
	\speak{USER} Sex on the beach
	\speak{ALEXA} Welcome to sex on the beach.
	\speak{USER} I want to make a cocktail without alcohol.
	\speak{ALEXA} Which ingredient do you want in your non alcoholic cocktail?
	\speak{USER} Milk.
	\speak{ALEXA} I know 10 cocktails with milk and no alcohol. I could give you three examples?
	\speak{USER} Yes, please.
	\speak{ALEXA} Banana Milk Shake, Sweet Bananas, Frappé.
	\speak{USER} Give me the ingredients for a frappe.
	\speak{ALEXA} 3 ingredients are needed for a frappe: Coffee, Milk, Sugar.
	\speak{USER} How much sugar?
	\speak{ALEXA} You need 1-2 tsp sugar for a frappe.
	\speak{USER} Which glass do i need?
	\speak{ALEXA} For a frappe a Highball Glass is the right choice.
	\speak{USER} What are the instructions?
	\speak{ALEXA} The instructions for frappe are the following: You need 1/2 cup black Coffee, 1/2 cup Milk, 1-2 tsp Sugar. Mix together. Blend at highest blender speed for about 1 minute. Pour into a glass and drink with a straw. Notes: This works best if everything is cold (if you make fresh coffee, mix it with the milk and let it sit in the fridge for 1/2 hour. If it is not frothy, add more milk, or even just some more milk powder. The froth gradually turns to liquid at the bottom of the glass, so you will find that you can sit and drink this for about 1/2 hour, with more iced coffee continually appearing at the bottom. Very refreshing.
	
\end{dialogue}

\section{Projektorganisation}
\subsection{Aufgabenverteilung}
\subsection{Planungsdokumentation}

\section{Entwurf des Systems, Dokumentation}
\subsection{Entwicklungsumgebung, Dokumentation der Intents}

\subsection{AskForCocktailIntent}
Der Intent gibt für einen Cocktail die Zutaten, die Anleitung oder beides aus.
Der Intent hat zwei slots, den slot 'request' und den slot 'drink'. 'request' muss obligatorisch gefüllt werden. Im slot 'request' geben Benutzer*Innen an ob sie Zutaten, Anleitung oder beides hören möchten, dementsprechend wird ein API-request gestellt. 'drink' ist ein optionaler slot. Der Skill arbeitet mit session attributes, Im randomcocktailintent beispielsweise wird der vorgeschlagene Cocktail zum session attribut. Dass 'drink' optional ist soll ermöglichen, dass nicht für jede Anfrage der cocktail erneut genannt werden muss. Ist der 'drink' slot nicht gefüllt wird geprüft ob bereits ein Cocktail als session attribut existiert, aus vorher aufgerufenen Intents, dann wird die Anfrage für diesen Cocktail gemacht. Existert kein session attribute und der slot ist nicht von Benutzer*Innen gefüllt wird der slot gefüllt, indem aus der lambda funktion eine Anfrage an Benutzer*Innen gestellt wird. Dieser cocktail wird dann zum session atribute. Wird z.B aus dem random cocktail Intent in den Ask for cocktail intent gewechselt, dadurch dass Benutzer*Innen die Zutaten für den vorgschlagenen Cockteil wissen möchten, ist dieser session attribute und muss nicht erneut genannt werden.
Hat der Cocktail keinen Eintrag in der datenbank werden Benutzer*Innen darüber informiert.
\subsection{RandomCocktailIntent}
\subsection{IngredientDescriptionIntent}
Der Intent gibt eine kurze Beschreibung einer Zutat aus. Der Intent hat einen slot 'ingredient\_drink', dieser ist obligatorisch. Hat die Zutat keinen Eintrag in der Datenbank werden die Benutzer*Innen darüber informiert. Die Beschreibungstexte in der Datembank sind teilweise sehr lang. Aus dem Grund wird der Text in Sätze aufgesplittet und nur maximal die ersten drei Sätze werden als Antwort ausgegeben.
\subsection{CocktailWithIngredientIntent}
\subsection{NonAlcoholicCocktailIntent}
\subsection{GlassIntent}
Der Intent gibt eine kurze Information darüber aus in welchem Glas ein Cocktail üblicherweise serviert wird. Der Intent hat zwei slots die beide nicht obligtaorisch sind. Der Intent arbeitet ebenfalls mit session atributes, ist der slot 'drink' nicht gefüllt wird auf das session attribute zurückgrgriffen, falls benutzerinnen gerade aus einem Anderen Intent kommen, beispielsweise dem askforcocktailIntent ist dieses bereits gefüllt. Ist der slot nicht gefüllt und es existert kein session attribute wird der slot gefüllt, indem aus der lambda funktion eine Anfrage an Benutzer*Innen gestellt wird. Dieser Cocktail wird dann zum session attribute. Hat die datenbank die gewünschten Informatinen nicht werden die Benutzer*Innen darüber informiert.
\subsection{MeasureIntent}

\section{Projektabschluss, Evaluation}
\subsection{Versuchsanordnung}
Für den Versuch haben wir uns dazu entschieden ein user manual zu schreiben welches den Testpersonen
bereitgestellt wird und haben uns gegen eine konkrete Aufgabenstellung an die Testpersonen geeinigt. Fpr das Users Manual haben wir uns entschieden um den Testpersonen einen Eindruck zu geben welche Informationen sie durch den Skill generell abrufen können. Dadurch soll vermieden werden, dass funktionalitäten wie das Erfragen einer Mengenangabe oder das Erfragen des Richtigen Glases gar nicht in Betracht gezogen wird. Allerdings bestand dadurch die Gefahr die Testpersonen bezüglich ihrer Art den Skill aufzurufen zu primen.
In der Evaluation der entstandenen Dialoge stellte sich herraus das, dies der fall war/nicht der fall war.
\\
 
\begin{adjustwidth}{1cm}{1cm}
\textbf{Users Manual}

We are happy you decided to use the sex on the beach Alexa skill. Our skill is designed around cocktails, we seek to provide you with all infos concerning cocktails you may need. To invoke the skill say "Open sex on the beach".
Our skill has many functionalities; You can ask for the ingredient list and instructions for a specific cocktail, and if you can’t decide which glas your cocktail looks best in, you can ask Alexa which glas a cocktail is served in. Furthermore if you want to have more infos on a specific ingredient you can also ask for infos on that. If you have ingredients you prefer in your cocktails, you can ask Alexa for cocktails that contain those, maybe she knows some recipes, maybe not. If you don’t have any inspiration for your cocktail party yet, you can also ask Alexa to suggest you a cocktail.
If you need a reminder on how to use the skill, you can just ask Alexa for help or usage. And last but not least, if you need to know how much gin you need in a gin and tonic, Alexa can handle that too.
\\
\textbf{Users Manual 2} 

We are happy you decided to use the 'Sex on the Beach' Alexa skill. Sex on the Beach is designed around cocktails. We seek to provide you with all information on cocktails you may need. To invoke the skill simply say "Open sex on the beach".
Our skill has many functionalities; You can ask for the ingredient list and instructions of specific cocktails. If you can’t decide which glas your cocktail looks best in, you can ask Alexa how to serve it. Furthermore if you want to know more about a specific ingredient, you can ask for information on that. If you prefer certain ingredients, you can ask Alexa for cocktails that contain those. Maybe she knows some recipes, maybe she doesn't. Also if you need inspiration for your next cocktail party, you can ask Alexa for some suggestions. Last but not least, if you want to know how much gin you need for a gin tonic, Sex on the Beach can handle that as well.
If you need a reminder on how to use the skill, don't hesitate to ask for help or usage information. \\ \\
Example requests you can ask our skill to get familiar:
\begin{itemize}

\item open sex on the beach.

\item give me a cocktail idea!

\item what is Kahlua?

\item what is the recipe for an Ipanema?

\item search for cocktails with gin and sugar.

\item which glass do I need for a White Russian?

\item How much milk do I need for a Sweet Bananas?
\end{itemize}

\end{adjustwidth}

Testpersonen:
    Karla, Carlotta

\subsection{Auswertung}
\textbf{Testperson Karla}
\begin{dialogue}
\speak{KARLA} sex on the beach

\speak{ALEXA} Welcome to Sex on the beach

\speak{KARLA} can you suggest me a cocktail

\speak{ALEXA} A Rum Toddy is always a good descision, Do you want to hear the ingredients list?

\speak{KARLA} yes

\speak{ALEXA} 4 ingredients are needed for a Rum Toddy: Rum, Powdered sugar, Lemon peel, Water.

\speak{KARLA} can you suggest something else

\speak{ALEXA} I am sorry, I can't handle this request.\\
\direct{Diese funktionalität ist nicht eingebaut, darum scheitert der request}

\speak{KARLA} do you have another suggestion

\speak{ALEXA} I am sorry, I can't handle this request.

\speak{KARLA} what glass is rum toddy certain

\speak{ALEXA} I don't know more about rum toddy certain i'm sorry, maybe google knows more than me.\\
\direct{scheitert warscheinlich aufgrund des speech to text Fehlers}

\speak{KARLA} what do i deed for a white russian

\speak{ALEXA} what do you want me to look up, instructions , ingredients or both?

\speak{KARLA} ingredients

\speak{ALEXA} You need 3 ingredients to make a white Russian: Vodka, Coffee liqueur, Light cream.

\speak{KARLA} can you also give me the instructions

\speak{ALEXA} The instructions for white Russian are the following: You need 2 oz Vodka, 1 oz Coffee liqueur, some Light cream. Pour vodka and coffee liqueur over ice cubes in an old-fashioned glass. Fill with light cream and serve.

\speak{KARLA} what can i do with lemon and rum

\speak{ALEXA} I know 1 cocktails with lemon and rum. Do You want to hear them?

\speak{KARLA} yes sure

\speak{ALEXA} Rum Cooler

\speak{KARLA} how do i make rumpler
\direct{speech to text Fehler, wird an dieser Stelle leider weiter propagiert, Alexa erkennt keinen neuen cocktail, der slot wird nicht gefüllt und dementsprechend wird angenommen, dass noch über das aktuelle session atribute gesprochen wird, dieses ist white russian. Rum Cooler ist an dieser stelle nicht zum session attribute gemacht wirden, da häufig eine Liste an passenden cocktails ausgegeben wird.}

\speak{ALEXA} what do you want me to look up, instructions , ingredients or both?

\speak{KARLA} instructions

\speak{ALEXA} The instructions for white Russian are the following: You need 2 oz Vodka, 1 oz Coffee liqueur, some Light cream. Pour vodka and coffee liqueur over ice cubes in an old-fashioned glass. Fill with light cream and serve.

\speak{KARLA} how do i make run cooler\\
\direct{Speech to text fehler}

I am sorry, I can't handle this request.

\speak{KARLA} what is rum

\speak{ALEXA} Rum is a distilled alcoholic beverage made from sugarcane byproducts, such as molasses, or directly from sugarcane juice, by a process of fermentation and distillation..The distillate, a clear liquid, is then usually aged in oak barrels.

\speak{KARLA} do you know pina colada\\
\direct{wird vom Interaktionsmodell nicht abgedekt diese Anfrage}

\speak{ALEXA} I am sorry, I can't handle this request.

\speak{KARLA} what glass do i need for pina colada\\
\direct{pina colada wird nicht als Wert für den drink slot erkannt, darum wird wieder auf das session attribute zurückgegriffen}

\speak{ALEXA} For a white Russian a Old-fashioned glass is the right choice

\speak{KARLA} what are the ingredients for king pina colada\\
\direct{speech to text Fehler}

\speak{ALEXA} I don't know anything about king Pina colada, I am sorry.\\
\direct{King Pina colada ist nicht in der Datenbank enthalten}

\speak{KARLA} what are the ingredients for pina colada

\speak{ALEXA} 3 ingredients are needed for a Pina colada: Light rum, Coconut milk, Pineapple.

\speak{KARLA} how much coconut milk do i need for a nikolai\\
\direct{Speech to text Fehler}

\speak{ALEXA} I am sorry, something went wrong.
\speak{KARLA} Bye
\speak{ALEXA} Bye, have a nice day.
\end{dialogue}

Der Dialog wurde oft durch speech to text fehler gestört, die teils schlechte spracherkennung ist zum Teil wohl auch dem Umstand geschuldet, dass die Testperson keine Englischemuttersprachlerin ist.
Bis auf einen Fehler sind alle durch speech to text getriggert worden.
Eine wichtige Verbesserung wäre den RandomCocktailIntent dadurch zu erweitern dass er durch Anfragen wie "sugges me another" ebenfalls getriggert wird. Das lässt sich villt relativ einfach im Sprachmodell lösen.
Eine weitere Verbesserung die in der Lambda Funktion passieren muss; Wenn Alexa bei der Suche nach Cocktails nur einen findet anstatt mehrerer, diesen gefundenen Cocktail zum session attribut machen. \\

\textbf{Testperson Carlotta}
\begin{dialogue}
	\speak{CARLOTTA} sex on the beach
	\speak{ALEXA} Welcome to Sex on the beach
	\speak{CARLOTTA} what kind of cocktail should i make myself
	\speak{ALEXA} I am sorry, I can't handle this request. \\
	\direct{Anfrage wird nicht vom Interaktionsmodell abgedeckt. Wäre Zuständigkeit des RandomCocktailIntents.}
	\attrib{1}
	\speak{CARLOTTA} what should i drink
	\speak{ALEXA} I think a Lemon Drop sounds good. Do you want me to read you the ingredients?
	\speak{CARLOTTA} yes
	\speak{ALEXA} 3 ingredients are needed for a Lemon Drop: Absolut Vodka, Cointreau, Lemon.
	\speak{CARLOTTA} what's the measurement of the
	\speak{ALEXA} I am sorry, I can't handle this request. \\
	\direct{Speech-to-Text Fehler, aber die Anfrage wird auch nicht vom Interaktionsmodell abgedenkt. Wäre Zuständigkeit des MeasureIntents.}
	\attrib{2}
	\speak{CARLOTTA} how do i mix a lemon drop
	\speak{ALEXA} I am sorry, I can't handle this request. \\
	\direct{Anfrage wird nicht vom Interaktionsmodell abgedeckt. Wäre Zuständigkeit des AskForCocktailIntents.}
	\attrib{3}
	\speak{CARLOTTA} how do i surf a serve and whiskey sour
	\speak{ALEXA} For a Lemon Drop a Cocktail glass is the right choice \\
	\direct{Aufgrund der Formulierung ist nicht klar, ob das Glas oder das Rezept für den Cocktail erfragt wird. Es wird der GlassIntent getriggert, aber der Cocktail nicht erkannt, somit wird der vorherige 'aktuelle' Cocktail aus den Session Attributen verwendet. Könnte mit dem Speech-to-Text Fehler zusammenhängen.}
	\attrib{4}
	\speak{CARLOTTA} what other what other kinds of cocktails are served in an cocktail
	\speak{ALEXA} I am sorry, I can't handle this request. \\
	\direct{Die gewünschte Funktionalität ist nicht eingebaut. Der Speech-to-Text Fehler hat somit keine Auswirkungen.}
	\attrib{5}
	\speak{CARLOTTA} what's in an moscow mule
	\speak{ALEXA} You can search for cocktails by ingredients or by name. \\
	\direct{Request scheitert aufgrund des Speech-to-Text Fehlers}
	\attrib{6}
	\speak{CARLOTTA} look for moscow mule
	\speak{ALEXA} what do you want me to look up, instructions , ingredients or both?
	\speak{CARLOTTA} both please
	\speak{ALEXA} Here are the instructions for making a Moscow mule: You need 2 oz Vodka, 2 oz Lime juice, 8 oz Ginger ale. Combine vodka and ginger beer in a highball glass filled with ice. Add lime juice. Stir gently. Garnish.
	\speak{CARLOTTA} what kind of cocktail do i serve my friends
	\speak{ALEXA} Moscow mule is served in a Copper Mug. \\
	\direct{Anfrage wird nicht vom Interaktionsmodell abgedeckt. Wäre Zuständigkeit des RandomCocktailIntents, aber der GlassIntent wird durch die Formulierung getriggert.}
	\attrib{7}
	\speak{CARLOTTA} what's kind of cocktail should i make tonight
	\speak{ALEXA} I am sorry, I can't handle this request. \\
	\direct{Anfrage wird nicht vom Interaktionsmodell abgedeckt. Wäre Zuständigkeit des RandomCocktailIntents.}
	\attrib{8}
	\speak{CARLOTTA} can you suggest an cocktail for me please
	\speak{ALEXA} I am sorry, I can't handle this request. \\
	\direct{Request scheitert aufgrund des Speech-to-Text Fehlers}
	\attrib{9}
	\speak{CARLOTTA} in what's kinds of cocktails is vodka
	\speak{ALEXA} You can search for cocktails by ingredients or by name.\\
	\direct{Anfrage wird nicht vom Interaktionsmodell abgedeckt. Wäre Zuständigkeit des CocktailWithIngredientIntents.}
	\attrib{10}
	\speak{CARLOTTA} how do i make a pina colada cocktail
	\speak{ALEXA} what do you want me to look up, instructions , ingredients or both?
	\speak{CARLOTTA} instructions please
	\speak{ALEXA} I am sorry, I don't have any information about Pina colada cocktail \\
	\direct{Wegen des Interaktionsmodells wird der Cocktailname in der Anfrage falsch interpretiert, dadurch wird nicht die richtige API-Anfrage gestellt.}
	\attrib{11}
	\speak{CARLOTTA} what's a easy cocktail to make
	\speak{ALEXA} what do you want me to look up, instructions , ingredients or both? \\
	\direct{Die gewünschte Funktionalität ist nicht eingebaut, bzw. wird die Anfrage nicht vom Interaktionsmodell abgedeckt, wenn man die Zuständigkeit dem RandomCocktailIntent zuordnet. Aufgrund der Formulierung wird der AskForCocktailIntent getriggert.}
	\attrib{12}
	\speak{CARLOTTA} both
	\speak{ALEXA} I have never heard of a easy cocktail, sorry 
	\speak{CARLOTTA} name your name your favorite cocktail
	\speak{ALEXA} You can't do anython wrong with a Z********d Do you want to know the ingredients?
	\speak{CARLOTTA} yes please
	\speak{ALEXA} You need 3 ingredients to make a Z********d: Chambord raspberry liqueur, Vodka, Soda water.
	\speak{CARLOTTA} what kind of cocktail is best for a party
	\speak{ALEXA} I am sorry, something went wrong. \\
	\direct{Die gewünschte Funktionalität ist nicht eingebaut, bzw. wird die Anfrage nicht vom Interaktionsmodell abgedeckt, wenn man die Zuständigkeit dem RandomCocktailIntent zuordnet.}
	\attrib{13}
	\speak{CARLOTTA} how much vodka is in a moscow mule
	\speak{ALEXA} You need 2 oz vodka for a Moscow mule.
	\speak{CARLOTTA} how do i make a diem
	\speak{ALEXA} what do you want me to look up, instructions , ingredients or both?
	\speak{CARLOTTA} ingredients
	\speak{ALEXA} I don't know anything about diem, I am sorry. \\
	\direct{Speech-to-Text Fehler beim Cocktailnamen.}
	\attrib{14}
	\speak{CARLOTTA} how much ingredients do i need for a bloody mary for four pie
	\speak{ALEXA} I am sorry, something went wrong. \\
	\direct{Die gewünschte Funktionalität ist nicht eingebaut. Der Speech-to-Text Fehler hat somit keine Auswirkungen.}
	\attrib{15}
	\speak{CARLOTTA} how do i make an alcoholic cocktail
	\speak{ALEXA} Which ingredient do you want in your non alcoholic cocktail \\
	\direct{Trotz des Speech-to-Text Fehlers wird der richtige Intent getriggert.}
	\speak{CARLOTTA} orange juice
	\speak{ALEXA} There are 8 cocktails that have orange juice and no alcohol. Should I list the three best ones?
	\speak{CARLOTTA} yes
	\speak{ALEXA} Apello, Pysch Vitamin Light, Fruit Shake
	\speak{CARLOTTA} can you repeat
	\speak{ALEXA} I am sorry, I can't handle this request. \\
	\direct{Die gewünschte Funktionalität ist nicht eingebaut.}
	\attrib{16}
	\speak{CARLOTTA} quit
\end{dialogue}

Insgesamt werden 29 Requests (abgesehen vom Cancel und Launch Request) gestellt. Bei 16 Requests sind Fehler aufgetreten. Vier von ihnen (4, 6, 9, 14) sind durch Schwierigkeiten der Spracherkennung entstanden und somit nicht von uns behebbar. Für neun der Fehler (1,2,3,7,8,10,11,12,13) ist das Interaktionsmodell verantwortlich, welches wir zur Behebung problemlos erweitern können sollten. In fünf Fällen kann die Anfrage nicht richtig bearbeitet werden, weil die gewünschte Funktion nicht implementiert ist. Dazu gehört das Filtern von Cocktails nach Gläsern (5), die Suche nach 'einfachen' Cocktails (12), die Suche nach Cocktails für bestimmte Anlässe (13), die Berechnung der Zutatenmenge nach Anzahl der Personen (15) und das Wiederholen des letzten Speech-Outputs (16).
\subsection{Ausblick}
Code besser strukturieren, immer noch zu viel doppelter code\\
mehr mit slot to elicit bzw intent changing von lambda aus arbeiten\\
mehrsprachig, - schwierig wegen datenbank fehler in Übersetzung.\\
Weitere Funktionen implementieren: Fragen nach Alkoholgehalt beantworten, Cocktailfilter nach Glas, Cocktails suchen nach verschiedenen Kriterien (z.B. schweregrad bzw. Anzahl der Zutaten) oder Anlässen, Berechnung der benötigten Zutatenmengen für verschiedene Personenanzahlen.
\end{document}